%!TEX root=../main.tex
\section{Ideas} % (fold)
\label{sec:ideas}

\begin{itemize}
    \item This should be a quick paper on the selection of the number of components for MFPCA.
    \item Based on how the number of components is selected in MFPCA \cite{happMultivariateFunctionalPrincipal2018a}.
    \item Just considering the number of components, based on an univariate expansion, we speculate that we need for that say $K$ univariate components to effectively estimate $K$ multivariate components.
    Let $K_p$ be the number of estimated components for the $p$th feature and $K$ the number of multivariate components we want to estimate. Computationally speaking, we can estimate up to $\sum_{p = 1}^P K_p$ multivariate components. We however claim that the number of accurately estimated components is only $\min_{p = 1, \dots, P} K_p$.
    \item The same phenomenon appears with the percentage of variance explained, we can not retrieve $\alpha\%$ of the variance with the multivariate curves, if the univariate components also explained $\alpha\%$ of the univariate curves. This might be related to the ``multivariate testing'' phenomena. Mentioned in the Supplementary material of \cite{happMultivariateFunctionalPrincipal2018a}. We rerun the sensitivity analysis and aim to show that the percentage of variance explained might be different of one expects.
    \item This is going to be a practical paper, no proof, except pratical proofs will be presented here. Only work on extensive simulations.
\end{itemize}

% section ideas (end)