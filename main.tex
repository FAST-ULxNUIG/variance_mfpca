\documentclass{article}

\usepackage[english]{babel}
\usepackage[a4paper,top=2cm,bottom=2cm,left=3cm,right=3cm,marginparwidth=1.75cm]{geometry}

\usepackage{natbib}
\usepackage{amsmath}
\usepackage{amssymb}
\usepackage{amsthm}
\usepackage{mathtools}\mathtoolsset{showonlyrefs=true}
\usepackage{graphicx}
\usepackage[colorlinks=true, allcolors=blue]{hyperref}

\usepackage{caption}
\usepackage{subcaption}

\usepackage{dsfont}

% -------------
% Some commands
% -------------

% Environment
\newcounter{rem}
\newtheorem{remark}[rem]{Remark}

\newcounter{th}
\newtheorem{theorem}[th]{Theorem}

\newcounter{scenario}[section]
\newenvironment{scenario}[1][]{\refstepcounter{scenario}\par\medskip
   \noindent \textbf{Scenario~\thescenario. #1} \rmfamily}{\medskip}

\newenvironment{results}[1][]{\noindent \textbf{#1} \rmfamily}{\medskip}

\providecommand{\keywords}[1]{\textbf{\textit{Keywords---}} #1}

% Stats
\newcommand{\EE}{\mathbb{E}} % Expectation
\newcommand{\RR}{\mathbb{R}} % R
\newcommand{\NN}{\mathbb{N}} % N
\newcommand{\XX}{\mathcal{X}} % set X
\newcommand{\dd}{{\rm d}}

% Spaces
\newcommand{\TT}[1]{\mathcal{T}_{#1}} % Domain definition space
\newcommand{\sLp}[1]{\mathcal{L}^{2}\left(#1\right)} % L^p space
\newcommand{\HH}{\mathcal{H}} % Product of L^p space
\newcommand{\GG}{\mathcal{G}} % (L^p)^N space

% Inner product and norm
\newcommand{\pointt}{\mathbf{t}} % Indexed of multivariate curves
\newcommand{\points}{\mathbf{s}} % Indexed of multivariate curves
\newcommand{\inLp}[2]{\left\langle#1, #2\right\rangle} % Inner product in Lp
\newcommand{\inR}[2]{\left(#1, #2\right)}
\newcommand{\inRM}[2]{\left(#1, #2\right)_{\mathbf{M}}}
\newcommand{\normLp}[1]{\left|\!\left|#1\right|\!\right|} % Norm in Lp
\newcommand{\normR}[1]{\left(\!\left(#1\right)\!\right)} %
\newcommand{\normRM}[1]{\left(\!\left(#1\right)\!\right)_{\mathbf{M}}} %
\newcommand{\inH}[2]{\langle\!\langle#1, #2\rangle\!\rangle}
\newcommand{\inHw}[2]{\langle\!\langle#1, #2\rangle\!\rangle_w}
\newcommand{\inHG}[2]{\langle\!\langle#1, #2\rangle\!\rangle_\Gamma}
\newcommand{\normH}[1]{\left|\!\left|\!\left|#1\right|\!\right|\!\right|}
\newcommand{\normHG}[1]{\left|\!\left|\!\left|#1\right|\!\right|\!\right|_\Gamma}

% Data/Process related
\newcommand{\Xnp}{X_n^{(p)}} % Observation n, feature p
\newcommand{\Xnq}{X_n^{(q)}} % Observation n, feature q
\newcommand{\hatXnp}{\widehat{X}_n^{(p)}} % Reconstruction
\newcommand{\Xp}[1]{X^{(#1)}} % Feature p
\newcommand{\mup}[1]{\mu^{(#1)}} % Feature p
\newcommand{\fp}{f^{(p)}} 
\newcommand{\gp}{g^{(p)}}

% Geometric related
\newcommand{\pobs}[1]{\mathrm{#1}} % Point related to observations
\newcommand{\CN}{\mathcal{C}_{\!N}} % Cloud of features
\newcommand{\Gmu}{\pobs{G}_{\!\mu}} % Centre of gravity of C_N
\newcommand{\OH}{\pobs{O}_{\!\mathcal{H}}} % Centre of H

\newcommand{\pfea}[1]{\mathsf{#1}} % Point related to features
\newcommand{\CP}{\mathcal{C}_{\!P}} % Cloud of observations
\newcommand{\Gfea}{\pfea{G}_{\!\mu}} % Centre of gravity of C_P
\newcommand{\OG}{\pfea{O}_{\!\RR}} % Centre of H

% Declaration of math operator
\DeclareMathOperator{\Var}{Var}
\DeclareMathOperator{\Cov}{Cov}
\DeclareMathOperator*{\argmax}{arg\,max}
\DeclareMathOperator{\bigO}{\mathcal{O}}

% Function subset restriction  
\newcommand\restr[2]{{ %
  \left.\kern-\nulldelimiterspace  %
  #1  %
  \vphantom{\big|}  %
  \right|_{#2}  %
}}
% -------------

\title{A note on the number of components retained for multivariate functional principal components analysis}
\author{Steven Golovkine}
\date{\today}

\begin{document}
\maketitle

\begin{abstract}
Your abstract.
\end{abstract}

% MAIN --------
%!TEX root=../main.tex
\section{Introduction} % (fold)
\label{sec:introduction}

We aim to show that the procedure proposed by \cite{happMultivariateFunctionalPrincipal2018a} may leaad to inconsistency in the retained number of components, based on extensive simulation.

The simulation may varies has follow:
\begin{itemize}
    \item Number of curves $N = 25, 50, 100, 200$
    \item Number of sampling points $M = 25, 50, 100, 200$
    \item Number of components $P = 2, 5, 10, 20, 50$
    \item No noise and we assume the curve are sampled on a common grid.
    \item Based on the Karhunen-Loève decomposition, make sure that the decreasing of the eigenvalues is coherent with KL assumptions. The data are defined with a large number of components and different decreasing of the eigenvalues scenarios.
    \item We do the same for the percentage of variance explained. We set the percentage of variance explained for the multivariate components to be $\alpha\%$ ($\alpha = 50, 75, 90, 95, 99$) and we change the percentage of variance explained by the univariate components.
    \item The quality of the estimation is based on different measures: the number of retained components (only if we set the percentage of variance explained), the estimation of the eigenvalues ($\log-$AE), the estimation of the multivariate eigenfunctions (ISE) and the reconstruction of the curves (MISE).
    \item Data are simulated with \cite{happMultivariateFunctionalPrincipal2018a} setting and we can use ICHEC to run them.
\end{itemize}
% section introduction (end)

%!TEX root=../main.tex
\section{Ideas} % (fold)
\label{sec:ideas}

\begin{itemize}
    \item This should be a quick paper on the selection of the number of components for MFPCA.
    \item Based on how the number of components is selected in MFPCA \cite{happMultivariateFunctionalPrincipal2018a}.
    \item Just considering the number of components, based on an univariate expansion, we speculate that we need for that say $K$ univariate components to effectively estimate $K$ multivariate components.
    \item The same phenomenon appears with the percentage of variance explained, we can not retrieve $\alpha\%$ of the variance with the multivariate curves, if the univariate components also explained $\alpha\%$ of the univariate curves. This might be related to the ``multivariate testing'' phenomena.
    \item This is going to be a practical paper, no proof, except pratical proofs will be presented here. Only work on extensive simulations.
\end{itemize}

% section ideas (end)
% -------------


\bibliographystyle{abbrv}
\bibliography{./biblio.bib}

\end{document}

